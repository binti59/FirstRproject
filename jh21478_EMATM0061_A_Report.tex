% Options for packages loaded elsewhere
\PassOptionsToPackage{unicode}{hyperref}
\PassOptionsToPackage{hyphens}{url}
%
\documentclass[
]{article}
\usepackage{amsmath,amssymb}
\usepackage{lmodern}
\usepackage{ifxetex,ifluatex}
\ifnum 0\ifxetex 1\fi\ifluatex 1\fi=0 % if pdftex
  \usepackage[T1]{fontenc}
  \usepackage[utf8]{inputenc}
  \usepackage{textcomp} % provide euro and other symbols
\else % if luatex or xetex
  \usepackage{unicode-math}
  \defaultfontfeatures{Scale=MatchLowercase}
  \defaultfontfeatures[\rmfamily]{Ligatures=TeX,Scale=1}
\fi
% Use upquote if available, for straight quotes in verbatim environments
\IfFileExists{upquote.sty}{\usepackage{upquote}}{}
\IfFileExists{microtype.sty}{% use microtype if available
  \usepackage[]{microtype}
  \UseMicrotypeSet[protrusion]{basicmath} % disable protrusion for tt fonts
}{}
\makeatletter
\@ifundefined{KOMAClassName}{% if non-KOMA class
  \IfFileExists{parskip.sty}{%
    \usepackage{parskip}
  }{% else
    \setlength{\parindent}{0pt}
    \setlength{\parskip}{6pt plus 2pt minus 1pt}}
}{% if KOMA class
  \KOMAoptions{parskip=half}}
\makeatother
\usepackage{xcolor}
\IfFileExists{xurl.sty}{\usepackage{xurl}}{} % add URL line breaks if available
\IfFileExists{bookmark.sty}{\usepackage{bookmark}}{\usepackage{hyperref}}
\hypersetup{
  pdftitle={jh21478\_EMATM0061\_A\_Report},
  pdfauthor={Bikramjit Chowdhury},
  hidelinks,
  pdfcreator={LaTeX via pandoc}}
\urlstyle{same} % disable monospaced font for URLs
\usepackage[margin=1in]{geometry}
\usepackage{color}
\usepackage{fancyvrb}
\newcommand{\VerbBar}{|}
\newcommand{\VERB}{\Verb[commandchars=\\\{\}]}
\DefineVerbatimEnvironment{Highlighting}{Verbatim}{commandchars=\\\{\}}
% Add ',fontsize=\small' for more characters per line
\usepackage{framed}
\definecolor{shadecolor}{RGB}{248,248,248}
\newenvironment{Shaded}{\begin{snugshade}}{\end{snugshade}}
\newcommand{\AlertTok}[1]{\textcolor[rgb]{0.94,0.16,0.16}{#1}}
\newcommand{\AnnotationTok}[1]{\textcolor[rgb]{0.56,0.35,0.01}{\textbf{\textit{#1}}}}
\newcommand{\AttributeTok}[1]{\textcolor[rgb]{0.77,0.63,0.00}{#1}}
\newcommand{\BaseNTok}[1]{\textcolor[rgb]{0.00,0.00,0.81}{#1}}
\newcommand{\BuiltInTok}[1]{#1}
\newcommand{\CharTok}[1]{\textcolor[rgb]{0.31,0.60,0.02}{#1}}
\newcommand{\CommentTok}[1]{\textcolor[rgb]{0.56,0.35,0.01}{\textit{#1}}}
\newcommand{\CommentVarTok}[1]{\textcolor[rgb]{0.56,0.35,0.01}{\textbf{\textit{#1}}}}
\newcommand{\ConstantTok}[1]{\textcolor[rgb]{0.00,0.00,0.00}{#1}}
\newcommand{\ControlFlowTok}[1]{\textcolor[rgb]{0.13,0.29,0.53}{\textbf{#1}}}
\newcommand{\DataTypeTok}[1]{\textcolor[rgb]{0.13,0.29,0.53}{#1}}
\newcommand{\DecValTok}[1]{\textcolor[rgb]{0.00,0.00,0.81}{#1}}
\newcommand{\DocumentationTok}[1]{\textcolor[rgb]{0.56,0.35,0.01}{\textbf{\textit{#1}}}}
\newcommand{\ErrorTok}[1]{\textcolor[rgb]{0.64,0.00,0.00}{\textbf{#1}}}
\newcommand{\ExtensionTok}[1]{#1}
\newcommand{\FloatTok}[1]{\textcolor[rgb]{0.00,0.00,0.81}{#1}}
\newcommand{\FunctionTok}[1]{\textcolor[rgb]{0.00,0.00,0.00}{#1}}
\newcommand{\ImportTok}[1]{#1}
\newcommand{\InformationTok}[1]{\textcolor[rgb]{0.56,0.35,0.01}{\textbf{\textit{#1}}}}
\newcommand{\KeywordTok}[1]{\textcolor[rgb]{0.13,0.29,0.53}{\textbf{#1}}}
\newcommand{\NormalTok}[1]{#1}
\newcommand{\OperatorTok}[1]{\textcolor[rgb]{0.81,0.36,0.00}{\textbf{#1}}}
\newcommand{\OtherTok}[1]{\textcolor[rgb]{0.56,0.35,0.01}{#1}}
\newcommand{\PreprocessorTok}[1]{\textcolor[rgb]{0.56,0.35,0.01}{\textit{#1}}}
\newcommand{\RegionMarkerTok}[1]{#1}
\newcommand{\SpecialCharTok}[1]{\textcolor[rgb]{0.00,0.00,0.00}{#1}}
\newcommand{\SpecialStringTok}[1]{\textcolor[rgb]{0.31,0.60,0.02}{#1}}
\newcommand{\StringTok}[1]{\textcolor[rgb]{0.31,0.60,0.02}{#1}}
\newcommand{\VariableTok}[1]{\textcolor[rgb]{0.00,0.00,0.00}{#1}}
\newcommand{\VerbatimStringTok}[1]{\textcolor[rgb]{0.31,0.60,0.02}{#1}}
\newcommand{\WarningTok}[1]{\textcolor[rgb]{0.56,0.35,0.01}{\textbf{\textit{#1}}}}
\usepackage{graphicx}
\makeatletter
\def\maxwidth{\ifdim\Gin@nat@width>\linewidth\linewidth\else\Gin@nat@width\fi}
\def\maxheight{\ifdim\Gin@nat@height>\textheight\textheight\else\Gin@nat@height\fi}
\makeatother
% Scale images if necessary, so that they will not overflow the page
% margins by default, and it is still possible to overwrite the defaults
% using explicit options in \includegraphics[width, height, ...]{}
\setkeys{Gin}{width=\maxwidth,height=\maxheight,keepaspectratio}
% Set default figure placement to htbp
\makeatletter
\def\fps@figure{htbp}
\makeatother
\setlength{\emergencystretch}{3em} % prevent overfull lines
\providecommand{\tightlist}{%
  \setlength{\itemsep}{0pt}\setlength{\parskip}{0pt}}
\setcounter{secnumdepth}{-\maxdimen} % remove section numbering
\ifluatex
  \usepackage{selnolig}  % disable illegal ligatures
\fi

\title{jh21478\_EMATM0061\_A\_Report}
\author{Bikramjit Chowdhury}
\date{29/11/2021}

\begin{document}
\maketitle

\hypertarget{r-markdown}{%
\subsection{R Markdown}\label{r-markdown}}

This is an R Markdown document. Markdown is a simple formatting syntax
for authoring HTML, PDF, and MS Word documents. For more details on
using R Markdown see \url{http://rmarkdown.rstudio.com}.

When you click the \textbf{Knit} button a document will be generated
that includes both content as well as the output of any embedded R code
chunks within the document. You can embed an R code chunk like this:

\begin{Shaded}
\begin{Highlighting}[]
\DocumentationTok{\#\#install.packages("tidyverse")}
\DocumentationTok{\#\#install.packages("caret")}
\FunctionTok{library}\NormalTok{(tidyverse)}
\end{Highlighting}
\end{Shaded}

\begin{verbatim}
## -- Attaching packages --------------------------------------- tidyverse 1.3.1 --
\end{verbatim}

\begin{verbatim}
## v ggplot2 3.3.5     v purrr   0.3.4
## v tibble  3.1.4     v dplyr   1.0.7
## v tidyr   1.1.4     v stringr 1.4.0
## v readr   2.0.2     v forcats 0.5.1
\end{verbatim}

\begin{verbatim}
## -- Conflicts ------------------------------------------ tidyverse_conflicts() --
## x dplyr::filter() masks stats::filter()
## x dplyr::lag()    masks stats::lag()
\end{verbatim}

\begin{Shaded}
\begin{Highlighting}[]
\DocumentationTok{\#\#install.packages("Stat2Data")}
\FunctionTok{library}\NormalTok{(Stat2Data)}
\FunctionTok{library}\NormalTok{(readxl) }\CommentTok{\#load the readxl library}
\FunctionTok{library}\NormalTok{(ggplot2)}
\FunctionTok{library}\NormalTok{(glmnet)}
\end{Highlighting}
\end{Shaded}

\begin{verbatim}
## Loading required package: Matrix
\end{verbatim}

\begin{verbatim}
## 
## Attaching package: 'Matrix'
\end{verbatim}

\begin{verbatim}
## The following objects are masked from 'package:tidyr':
## 
##     expand, pack, unpack
\end{verbatim}

\begin{verbatim}
## Loaded glmnet 4.1-3
\end{verbatim}

\begin{Shaded}
\begin{Highlighting}[]
\FunctionTok{library}\NormalTok{(stringr)}
\FunctionTok{library}\NormalTok{(caret)}
\end{Highlighting}
\end{Shaded}

\begin{verbatim}
## Loading required package: lattice
\end{verbatim}

\begin{verbatim}
## 
## Attaching package: 'caret'
\end{verbatim}

\begin{verbatim}
## The following object is masked from 'package:purrr':
## 
##     lift
\end{verbatim}

\begin{Shaded}
\begin{Highlighting}[]
\CommentTok{\#Section A}
\CommentTok{\#Locate the folder path to read the file}
\NormalTok{folder\_path}\OtherTok{\textless{}{-}}\StringTok{"/Users/bikramjitchowdhury/Downloads/SCEM/jh21478\_EMATM0061\_summative\_assessment/jh21478\_EMATM0061\_A/"}
\CommentTok{\#Define the file name}
\NormalTok{file\_name}\OtherTok{\textless{}{-}}\StringTok{"finance\_data\_EMATM0061.csv"} \CommentTok{\#}
\CommentTok{\#Define the file path}
\NormalTok{file\_path}\OtherTok{\textless{}{-}}\FunctionTok{paste}\NormalTok{(folder\_path,file\_name,}\AttributeTok{sep=}\StringTok{""}\NormalTok{) }\CommentTok{\# create the file\_path}
\CommentTok{\#Read the file and assign it to a new data frame finance\_data\_original }

\NormalTok{finance\_data\_original}\OtherTok{\textless{}{-}}\FunctionTok{read\_csv}\NormalTok{(file\_path) }
\end{Highlighting}
\end{Shaded}

\begin{verbatim}
## Rows: 1051 Columns: 30
\end{verbatim}

\begin{verbatim}
## -- Column specification --------------------------------------------------------
## Delimiter: ","
## chr  (1): state_year_code
## dbl (29): Totals.Capital.outlay, Totals.Revenue, Totals.Expenditure, Totals....
\end{verbatim}

\begin{verbatim}
## 
## i Use `spec()` to retrieve the full column specification for this data.
## i Specify the column types or set `show_col_types = FALSE` to quiet this message.
\end{verbatim}

\begin{Shaded}
\begin{Highlighting}[]
\CommentTok{\#Answer A.1}
\CommentTok{\#Rows: 1051 Columns: 30 {-}There are 1051 rows and 30 columns in this file}
\end{Highlighting}
\end{Shaded}

\begin{Shaded}
\begin{Highlighting}[]
\CommentTok{\#Answer A.2}
\CommentTok{\#Calculate the number of columns in the original data frame}
\FunctionTok{ncol}\NormalTok{(finance\_data\_original) }\CommentTok{\# 30}
\end{Highlighting}
\end{Shaded}

\begin{verbatim}
## [1] 30
\end{verbatim}

\begin{Shaded}
\begin{Highlighting}[]
\CommentTok{\#Calculate the number of rows in the original data frame}
\FunctionTok{nrow}\NormalTok{(finance\_data\_original)  }\CommentTok{\#1051}
\end{Highlighting}
\end{Shaded}

\begin{verbatim}
## [1] 1051
\end{verbatim}

\begin{Shaded}
\begin{Highlighting}[]
\CommentTok{\#Create a subset of records called finance\_data with 5 rename columns but 1051 rows}
\NormalTok{finance\_data}\OtherTok{\textless{}{-}}\NormalTok{finance\_data\_original}\SpecialCharTok{\%\textgreater{}\%}\FunctionTok{select}\NormalTok{(state\_year\_code,Details.Education.Education.Total,Details.Health.Health.Total.Expenditure,Details.Transportation.Highways.Highways.Total.Expenditure,Totals.Revenue,Totals.Expenditure)}\SpecialCharTok{\%\textgreater{}\%}\FunctionTok{rename}\NormalTok{(}\AttributeTok{education\_expenditure=}\NormalTok{Details.Education.Education.Total,}\AttributeTok{health\_expenditure=}\NormalTok{Details.Health.Health.Total.Expenditure,}\AttributeTok{transport\_expenditure=}\NormalTok{Details.Transportation.Highways.Highways.Total.Expenditure,}\AttributeTok{totals\_revenue=}\NormalTok{Totals.Revenue,}\AttributeTok{totals\_expenditure=}\NormalTok{Totals.Expenditure) }


\CommentTok{\#Display the results for the first 5 rows and first 5 columns only}
\FunctionTok{head}\NormalTok{(finance\_data}\SpecialCharTok{\%\textgreater{}\%}\FunctionTok{select}\NormalTok{(state\_year\_code,education\_expenditure,health\_expenditure),}\DecValTok{5}\NormalTok{)}
\end{Highlighting}
\end{Shaded}

\begin{verbatim}
## # A tibble: 5 x 3
##   state_year_code education_expenditure health_expenditure
##   <chr>                           <dbl>              <dbl>
## 1 ALABAMA__1992                 3570524                 NA
## 2 ALABAMA__1993                 3663465                 NA
## 3 ALABAMA__1994                 3969277             487044
## 4 ALABAMA__1995                 4400912             491648
## 5 ALABAMA__1996                 4872259             514380
\end{verbatim}

\begin{Shaded}
\begin{Highlighting}[]
\CommentTok{\#Answer A.3}
\CommentTok{\#Add the new column  total\_savings in the finance data dataframe}
\NormalTok{finance\_data}\OtherTok{\textless{}{-}}\NormalTok{finance\_data}\SpecialCharTok{\%\textgreater{}\%}\FunctionTok{mutate}\NormalTok{(}\AttributeTok{totals\_savings=}\NormalTok{(totals\_revenue}\SpecialCharTok{{-}}\NormalTok{totals\_expenditure))}

\CommentTok{\#Display the first 3 rows and the four columns “state\_year\_code”,“totals\_revenue”,“totals\_expenditure”,“totals\_savings” of the data frame finance\_data}
\FunctionTok{head}\NormalTok{(finance\_data}\SpecialCharTok{\%\textgreater{}\%}\FunctionTok{select}\NormalTok{(state\_year\_code,totals\_revenue,totals\_expenditure,totals\_savings),}\DecValTok{3}\NormalTok{)}
\end{Highlighting}
\end{Shaded}

\begin{verbatim}
## # A tibble: 3 x 4
##   state_year_code totals_revenue totals_expenditure totals_savings
##   <chr>                    <dbl>              <dbl>          <dbl>
## 1 ALABAMA__1992         10536166            9650515         885651
## 2 ALABAMA__1993         11389335           10242374        1146961
## 3 ALABAMA__1994         11599362           10815221         784141
\end{verbatim}

\begin{Shaded}
\begin{Highlighting}[]
\CommentTok{\#A.4}


\CommentTok{\#Separate the state\_year\_code coloumn into two seperate columns}
\NormalTok{finance\_data}\OtherTok{\textless{}{-}}\NormalTok{finance\_data}\SpecialCharTok{\%\textgreater{}\%}\FunctionTok{separate}\NormalTok{(state\_year\_code,}\AttributeTok{into=}\FunctionTok{c}\NormalTok{(}\StringTok{"state"}\NormalTok{,}\StringTok{"year"}\NormalTok{),}\AttributeTok{sep=}\StringTok{"\_\_"}\NormalTok{)}

\CommentTok{\#Convert the states so that they appear with the first letter of each word in upper case and the remainder in lower case}

\NormalTok{finance\_data}\SpecialCharTok{$}\NormalTok{state}\OtherTok{=}\FunctionTok{str\_to\_title}\NormalTok{(finance\_data}\SpecialCharTok{$}\NormalTok{state)}

\CommentTok{\#Display the first 3 rows and the four columns}
\FunctionTok{head}\NormalTok{(finance\_data}\SpecialCharTok{\%\textgreater{}\%}\FunctionTok{select}\NormalTok{(state,year,totals\_revenue,totals\_expenditure,totals\_savings),}\DecValTok{3}\NormalTok{)}
\end{Highlighting}
\end{Shaded}

\begin{verbatim}
## # A tibble: 3 x 5
##   state   year  totals_revenue totals_expenditure totals_savings
##   <chr>   <chr>          <dbl>              <dbl>          <dbl>
## 1 Alabama 1992        10536166            9650515         885651
## 2 Alabama 1993        11389335           10242374        1146961
## 3 Alabama 1994        11599362           10815221         784141
\end{verbatim}

\begin{Shaded}
\begin{Highlighting}[]
\CommentTok{\#A.5}
\CommentTok{\#Generate a plot which displays the total revenue (“total\_revenue”) as function of the year (“year”) for the following}
\CommentTok{\#four states: Louisiana, Montana, Mississippi and Kentucky.}
\CommentTok{\#Display the revenue in terms of millions of dollars}

\NormalTok{finance\_data}\SpecialCharTok{\%\textgreater{}\%}\FunctionTok{rename}\NormalTok{(}\AttributeTok{State=}\NormalTok{state)}\SpecialCharTok{\%\textgreater{}\%}\FunctionTok{filter}\NormalTok{(State }\SpecialCharTok{\%in\%} \FunctionTok{c}\NormalTok{(}\StringTok{"Louisiana"}\NormalTok{,}\StringTok{"Montana"}\NormalTok{,}\StringTok{"Mississippi"}\NormalTok{,}\StringTok{"Kentucky"}\NormalTok{))}\SpecialCharTok{\%\textgreater{}\%}\FunctionTok{ggplot}\NormalTok{(}\FunctionTok{aes}\NormalTok{(}\AttributeTok{x=}\FunctionTok{as.numeric}\NormalTok{(year),}\AttributeTok{y=}\NormalTok{totals\_revenue}\SpecialCharTok{/}\DecValTok{1000000}\NormalTok{),}\AttributeTok{color=}\NormalTok{State)}\SpecialCharTok{+}\FunctionTok{xlab}\NormalTok{(}\StringTok{"Year"}\NormalTok{)}\SpecialCharTok{+}\FunctionTok{theme\_bw}\NormalTok{()}\SpecialCharTok{+}\FunctionTok{ylab}\NormalTok{(}\StringTok{"Revenue(in millions)"}\NormalTok{)}\SpecialCharTok{+}\FunctionTok{geom\_smooth}\NormalTok{(}\FunctionTok{aes}\NormalTok{(}\AttributeTok{color=}\NormalTok{State,}\AttributeTok{linetype=}\NormalTok{State),}\AttributeTok{size=}\DecValTok{1}\NormalTok{)}
\end{Highlighting}
\end{Shaded}

\begin{verbatim}
## `geom_smooth()` using method = 'loess' and formula 'y ~ x'
\end{verbatim}

\begin{verbatim}
## Warning: Removed 11 rows containing non-finite values (stat_smooth).
\end{verbatim}

\includegraphics{jh21478_EMATM0061_A_Report_files/figure-latex/unnamed-chunk-4-1.pdf}

\begin{Shaded}
\begin{Highlighting}[]
\CommentTok{\#A.6}

\CommentTok{\#Create a function called get\_decade() which takes as input a number and rounds that number down to the}
\CommentTok{\#nearest multiple of 10. }

\NormalTok{get\_decade}\OtherTok{\textless{}{-}}\ControlFlowTok{function}\NormalTok{(x)\{}
 
\NormalTok{  y}\OtherTok{=}\FunctionTok{floor}\NormalTok{((x}\SpecialCharTok{/}\DecValTok{10}\NormalTok{))}\SpecialCharTok{*}\DecValTok{10}\NormalTok{  ;}
  \FunctionTok{return}\NormalTok{(y);}
\NormalTok{\}}

\FunctionTok{get\_decade}\NormalTok{(}\DecValTok{243}\NormalTok{)}
\end{Highlighting}
\end{Shaded}

\begin{verbatim}
## [1] 240
\end{verbatim}

\begin{Shaded}
\begin{Highlighting}[]
\CommentTok{\#Use your get\_decade() function to add a new column to the “finance\_data” data frame called “decade” which should give the decade corresponding to the year column.}

\NormalTok{finance\_data}\OtherTok{\textless{}{-}}\NormalTok{finance\_data}\SpecialCharTok{\%\textgreater{}\%}\FunctionTok{mutate}\NormalTok{(}\AttributeTok{decade=}\FunctionTok{get\_decade}\NormalTok{(}\FunctionTok{as.numeric}\NormalTok{(year)))}

\CommentTok{\#Three states had the highest mean{-}average savings (“totals\_savings”) over the decade starting 2000}

\NormalTok{finance\_data}\SpecialCharTok{\%\textgreater{}\%}\FunctionTok{filter}\NormalTok{(decade}\SpecialCharTok{==}\DecValTok{2000}\NormalTok{)}\SpecialCharTok{\%\textgreater{}\%}\FunctionTok{group\_by}\NormalTok{(state)}\SpecialCharTok{\%\textgreater{}\%}\FunctionTok{summarise}\NormalTok{(}\AttributeTok{mean\_total\_savings=}\FunctionTok{mean}\NormalTok{(totals\_savings,}\AttributeTok{na.rm=}\ConstantTok{TRUE}\NormalTok{))}\SpecialCharTok{\%\textgreater{}\%}\FunctionTok{arrange}\NormalTok{(}\FunctionTok{desc}\NormalTok{(mean\_total\_savings))}\SpecialCharTok{\%\textgreater{}\%}\FunctionTok{head}\NormalTok{(}\DecValTok{3}\NormalTok{)}
\end{Highlighting}
\end{Shaded}

\begin{verbatim}
## # A tibble: 3 x 2
##   state      mean_total_savings
##   <chr>                   <dbl>
## 1 Texas                4103352.
## 2 Ohio                 3025980 
## 3 California           2510483.
\end{verbatim}

\begin{Shaded}
\begin{Highlighting}[]
\CommentTok{\#The three states with the highest mean total savings are Texas, Ohio and California.}

\CommentTok{\# Create the Alaska summary data frame with the following properties}
\CommentTok{\#(a) “decade” – the decade (1990, 2000, 2010)}
\CommentTok{\#(b) “ed\_mn” – the mean of the education expenditure in Alaska for the corresponding decade}
\CommentTok{\#(c) “ed\_md” – the median of the education expenditure in Alaska for the corresponding decade}
\CommentTok{\#(d) “he\_mn” – the mean of the health expenditure in Alaska for the corresponding decade}
\CommentTok{\#(e) “he\_md” – the median of the health expenditure in Alaska for the corresponding decade}
\CommentTok{\#(f) “tr\_mn” – the mean of the transport expenditure in Alaska for the corresponding decade}
\CommentTok{\#(g) “tr\_md” – the median of the transport expenditure in Alaska for the corresponding decade.}

\NormalTok{alaska\_summary}\OtherTok{\textless{}{-}}\NormalTok{finance\_data}\SpecialCharTok{\%\textgreater{}\%}\FunctionTok{filter}\NormalTok{(state}\SpecialCharTok{==}\StringTok{\textquotesingle{}Alaska\textquotesingle{}}\NormalTok{)}\SpecialCharTok{\%\textgreater{}\%}\FunctionTok{group\_by}\NormalTok{(decade)}\SpecialCharTok{\%\textgreater{}\%}\FunctionTok{select}\NormalTok{(decade,education\_expenditure,health\_expenditure,transport\_expenditure)}\SpecialCharTok{\%\textgreater{}\%}\FunctionTok{summarise}\NormalTok{(}\FunctionTok{across}\NormalTok{(}\FunctionTok{starts\_with}\NormalTok{(}\FunctionTok{c}\NormalTok{(}\StringTok{"education\_expenditure"}\NormalTok{,}\StringTok{"health\_expenditure"}\NormalTok{,}\StringTok{"transport\_expenditure"}\NormalTok{)),}\FunctionTok{list}\NormalTok{(}\AttributeTok{md=}\NormalTok{median,}\AttributeTok{mn=}\NormalTok{mean),}\AttributeTok{.names=}\StringTok{"\{substring(.col,1,2)\}\_\{.fn\}"}\NormalTok{))}

\CommentTok{\#Display the Alaska summary data frame}
\NormalTok{alaska\_summary}
\end{Highlighting}
\end{Shaded}

\begin{verbatim}
## # A tibble: 3 x 7
##   decade    ed_md    ed_mn  he_md   he_mn  tr_md   tr_mn
##    <dbl>    <dbl>    <dbl>  <dbl>   <dbl>  <dbl>   <dbl>
## 1   1990 1246864. 1246426.     NA     NA      NA     NA 
## 2   2000 1566815  1546233. 160168 152003. 687407 723347.
## 3   2010      NA       NA  278394 274688      NA     NA
\end{verbatim}

\begin{Shaded}
\begin{Highlighting}[]
\CommentTok{\#A.8 Create a function called impute\_by\_median which takes as input a vector numerical values, which may include some “NA”s, and replaces any missing values (“NA”s) with the median over the vector.}

\NormalTok{impute\_by\_median}\OtherTok{\textless{}{-}}\ControlFlowTok{function}\NormalTok{(x)\{}
\NormalTok{med }\OtherTok{\textless{}{-}}\FunctionTok{median}\NormalTok{(x,}\AttributeTok{na.rm=}\DecValTok{1}\NormalTok{)   }\CommentTok{\# first calculate the median of x}
\NormalTok{impute\_f}\OtherTok{\textless{}{-}}\ControlFlowTok{function}\NormalTok{(z)\{  }\CommentTok{\# coordinate wise imputation}
\ControlFlowTok{if}\NormalTok{(}\FunctionTok{is.na}\NormalTok{(z))\{}
\FunctionTok{return}\NormalTok{(med)}
\NormalTok{\}  }\CommentTok{\#if z is na replace with mean}
\ControlFlowTok{else}\NormalTok{\{}
\FunctionTok{return}\NormalTok{(z)}\CommentTok{\#otherwise leave in place}
\NormalTok{\} }
\NormalTok{\}}
\FunctionTok{return}\NormalTok{(}\FunctionTok{map\_dbl}\NormalTok{(x,impute\_f)) }\CommentTok{\#apply the map function to impute across vector}
\NormalTok{\}}

\CommentTok{\#generate a subset of your “finance\_data” data frame called “idaho\_2000” which contains all those rows in which the state column takes the value “Idaho” and the “decade” column takes the value “2000” and includes the columns “year”, “education\_expenditure”, “health\_expenditure”, “transport\_expenditure”, “totals\_revenue”,“totals\_expenditure”, “totals\_savings” (i.e. all columns except “state” and “decade”).}


\NormalTok{idaho\_2000}\OtherTok{\textless{}{-}}\NormalTok{finance\_data}\SpecialCharTok{\%\textgreater{}\%}\FunctionTok{filter}\NormalTok{(state}\SpecialCharTok{==}\StringTok{\textquotesingle{}Idaho\textquotesingle{}} \SpecialCharTok{\&}\NormalTok{ decade}\SpecialCharTok{==}\DecValTok{2000}\NormalTok{)}\SpecialCharTok{\%\textgreater{}\%}\FunctionTok{select}\NormalTok{(}\SpecialCharTok{{-}}\NormalTok{state,}\SpecialCharTok{{-}}\NormalTok{decade)}

\CommentTok{\#Now apply your “impute\_by\_median” data frame to create a new data frame called “idaho\_2000\_imputed” which is based on your existing “idaho\_2000” data frame but with any missing values replaced with the corresponding median value for the that column. }

\NormalTok{idaho\_2000\_imputed}\OtherTok{\textless{}{-}}\NormalTok{idaho\_2000}\SpecialCharTok{\%\textgreater{}\%}\FunctionTok{select}\NormalTok{(year,health\_expenditure,education\_expenditure,totals\_savings)}\SpecialCharTok{\%\textgreater{}\%}\FunctionTok{summarise}\NormalTok{(year,}\FunctionTok{across}\NormalTok{(}\FunctionTok{where}\NormalTok{(is.numeric),}\SpecialCharTok{\textasciitilde{}}\FunctionTok{impute\_by\_median}\NormalTok{(.x)))}

\CommentTok{\#Display the imputed values}
\NormalTok{idaho\_2000\_imputed}
\end{Highlighting}
\end{Shaded}

\begin{verbatim}
## # A tibble: 5 x 4
##   year  health_expenditure education_expenditure totals_savings
##   <chr>              <dbl>                 <dbl>          <dbl>
## 1 2000              98510                1663980        1083732
## 2 2001             103705                1768837         334438
## 3 2002             109896.               1768837        -746375
## 4 2003             116088                1886421         709085
## 5 2004             131165                1768837        1349740
\end{verbatim}

\end{document}
